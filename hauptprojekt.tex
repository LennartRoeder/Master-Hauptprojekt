% !TeX root = hauptprojekt.tex
% !TeX spellcheck = en_US

\documentclass[
,paper=a4
,twoside=false
,fontsize=11pt
,headsepline
,BCOR10mm
,DIV11
]{scrbook}
\usepackage[english, ngerman]{babel}
%% see http://www.tex.ac.uk/cgi-bin/texfaq2html?label=uselmfonts

% define \mysharp using the arev package
\usepackage{arev}
\newsavebox{\sharpbox}
\sbox{\sharpbox}{$\sharp$}
\newcommand{\mysharp}{\usebox{\sharpbox}}

\usepackage[T1]{fontenc}
\usepackage[utf8]{inputenc}
\usepackage{libertine}
\usepackage{pifont}
\usepackage{microtype}
\usepackage{textcomp}
\usepackage[english,refpage]{nomencl}
\usepackage{setspace}
\usepackage{makeidx}
\usepackage{listings}
\usepackage{natbib}
\usepackage[english,colorlinks=true]{hyperref}
\usepackage{soul}
\usepackage{hawstyle}
\usepackage{float}
\usepackage{csquotes}
\usepackage{tabularx}
\usepackage{subcaption}
\usepackage{usecases}
\usepackage{enumitem}
\usepackage{pgfplots}
\usepackage{siunitx}

% Initializing the functional requirements
\newcommand{\reqinitF}{
	% Create a new counter for keeping track of the last number
	\newcounter{reqcountbackupF}
	% Create a new counter for the custom label
	\newcounter{reqcountF}
	% Redefine the command for the last counter so when it is called
	% it prints the number like this in a bold font: R<number>
	\renewcommand{\thereqcountF}{\textbf{F\arabic{reqcountF}}}
}

% Used to define the start of the requirements
\newcommand{\reqstartF}{
	% Indicate the start of a new list and tell it to use the redefined
	% command and corresponding counter for every item
	\begin{list}{\thereqcountF}{\usecounter{reqcountF}}
		% Important part: set the value of the used counter to the
		% same value of the backup counter.
		\setcounter{reqcountF}{\value{reqcountbackupF}}
	}
	
	% Used to define the end of the requirements
	\newcommand{\reqendF}{
		% Important part: take the value of the used counter (after
		% being incremented by the requirement items) and store it
		% in the backup counter.
		\setcounter{reqcountbackupF}{\value{reqcountF}}
		% Mark the end of the list environment
	\end{list}
}

% Initializing the non-functional requirements
\newcommand{\reqinitNF}{
	\newcounter{reqcountbackupNF}
	\newcounter{reqcountNF}
	\renewcommand{\thereqcountNF}{\textbf{NF\arabic{reqcountNF}}}
}

\newcommand{\reqstartNF}{
	\begin{list}{\thereqcountNF}{\usecounter{reqcountNF}}
		\setcounter{reqcountNF}{\value{reqcountbackupNF}}
	}
	
	\newcommand{\reqendNF}{
		\setcounter{reqcountbackupNF}{\value{reqcountNF}}
	\end{list}
}

%% define some colors
\colorlet{BackgroundColor}{gray!20}
\colorlet{KeywordColor}{blue}
\colorlet{CommentColor}{black!60}
%% for tables
\colorlet{HeadColor}{gray!60}
\colorlet{Color1}{blue!10}
\colorlet{Color2}{white}

%% configure colors
\HAWifprinter{
	\colorlet{BackgroundColor}{gray!20}
	\colorlet{KeywordColor}{black}
	\colorlet{CommentColor}{gray}
	% for tables
	\colorlet{HeadColor}{gray!60}
	\colorlet{Color1}{gray!40}
	\colorlet{Color2}{white}
}{}
\lstset{%
	numbers=left,
	numberstyle=\tiny,
	stepnumber=1,
	numbersep=5pt,
	basicstyle=\ttfamily\small,
	keywordstyle=\color{KeywordColor}\bfseries,
	identifierstyle=\color{black},
	commentstyle=\color{CommentColor},
	backgroundcolor=\color{BackgroundColor},
	captionpos=b,
	fontadjust=true
}
\lstset{escapeinside={(*@}{@*)}, % used to enter latex code inside listings
	morekeywords={uint32_t, int32_t}
}
\ifpdfoutput{
	\hypersetup{bookmarksopen=false,bookmarksnumbered,linktocpage}
}{}

\clubpenalty=10000
\widowpenalty=10000
\displaywidowpenalty=10000

% unknown hyphenations
%\hyphenation{
%}

%% recalculate text area
\typearea[current]{last}

\makeindex
\makenomenclature

\begin{document}
	\selectlanguage{english}
	
	
	%%%%%
	%% customize (see readme.pdf for supported values)
	\HAWThesisProperties{Author={Lennart Karsten}
		,Title={Optimizing geospatial read-performance\\inside a multi-agent simulation system}
%		,EnglishTitle={Optimizing geospatial read-performance inside a multi-agent simulation system}
		,ThesisType={Hauptprojekt}
		,ExaminationType={Masterprüfung}
		,DegreeProgramme={Master Informatik}
		,ThesisExperts={Prof. Dr. Thomas Clemen}
		,ReleaseDate={\today}
	}

	%% title
	\frontmatter
	
	%% output title page
	\maketitle
	
	\onehalfspacing
	
	%% add abstract pages
	%% note: this is one command on multiple lines
	\HAWAbstractPage
	%% German abstract
	{GIS, performance, Multi-agent simulation}%
	{This work evaluates and suggests improvements to the geospatial performance of a multi-agent simulation system. The suggestions are evaluated and performance tested and the results allow GIS to be used for large scale simulations.\\}
	%% English abstract
%	{GIS, performance, Multi-agent simulation}%
%	{english abstract.}
	
	\newpage
	\singlespacing
	
	\tableofcontents
	\newpage
	%% enable if these lists should be shown on their own page
	%\listoftables
	%\listoffigures
	%\lstlistoflistings
	
	%% main
	\mainmatter
	\onehalfspacing
	% write to the log/stdout
	\typeout{===== File: chapter 1}

	% init requirements	
	\reqinitF
	\reqinitNF

	% !TeX spellcheck = en_US

\chapter{Introduction}
Geospatial data plays a major roll in the processing big data and in the age of mobile geo-referenced data is becoming increasingly relevant \citep{Lee2015, Kitchin2013, Graham2013}. With growing relevance of such big data, the efficient storage and manipulation is a crucial topic.\\
MARS LIFE provides a large scale simulation environment for multi-agent simulations. Most models rely on geo-referenced data, making it a natural match for GIS. The possibility to use it as input for simulations is therefor very desirable.\\
This work focuses on improving the LIFE simulation system by adding layers that allows to take advantage of raster and vector GIS capabilities.



\section{Motivation}
MARS LIFE used to have capabilities for using GIS. However, they did not meet the requirements in terms of performance and usability. For this reason they were never used in production. With a recent infrastructural change, it has been abandoned due to incompatibilities.\\
The incapability to use GIS created the necessary for workarounds to use geo-referenced data. While these solutions work, they cover a very narrow use-case tailored to the needs of a specific model. This contradicts the idea of a general purpose simulation system.\\
For the reasons mentioned above this work focuses on re-instantiating GIS support into MARS LIFE. This is done with the latest requirements for usability and performance in mind. The main focus is, to achieve a performance that can handle agent counts in the millions within a reasonable time.\\
To demonstrate the capabilities of this layers, they will be integrated into the simulation platform. The main model will then use it for it's obstacle detection during agent movement.
 % Introduction
	% !TeX spellcheck = en_US

% GIS Grundlagen
% GIS Libs & Software
% Datenbanken
% MARS Architektur + Rahmenbedingungen
\chapter{Basics}
This chapter elaborates on the fundamental concepts and technologies necessary to understand the following chapters. They consist of a general overview of Geographic Information Systems, the mentioned geospatial libraries and databases, as well as the MARS Eco-System and it's circumstances



\section{Geographic Information System (GIS)}
GIS consists of numerous technologies to store, manipulate, analyze and visualize geographical data. It can be used in all domains that require the use of temporal and spatial data. Some usages are the visualization of land-use, elevation data, weather maps, street networks and flood maps. To leverage the capabilities that GIS offers, specialized software is required that can handle the spatial references, such as ESRI ArcMap QGIS or GrassGIS.


\subsection{Coordinate Reference System (CRS)}
In comparison to normal image data, such as JPEG or PNG, GIS data is geo-referenced, meaning each feature or pixel has a geospatial location. This is achieved by specific geo-aware formats that encode spatial positions into the data. This location is represented in a coordinate system. In GIS terminology this is referred to as \enquote{spatial reference system} (SRS) or \enquote{coordinate reference system} (CRS). Depending on the spatial area, coordinate systems provide different accuracy in their results. Some are optimized for certain areas, while others offer a general worldwide accuracy. The following sections explain common reference systems.

\subsubsection{Mercator Projection}
The most globally used CRS is the Mercator projection. I was created by Gerardus Mercator in 1569 \cite{meer2012atlas} and has been improved over the years. Figure \ref{fig:mercator} shows the Mercator projection in it's normal (left) and transverse (right) orientation. The normal projection offers good general representation, while the transverse orientation is focuses on the poles.
\begin{figure}[H]
	\centering\includegraphics[width=1\textwidth]{res/Mercator}
	\caption{Normal and transverse Mercator projection. \url{https://commons.wikimedia.org/w/index.php?curid=9910866}}
	\label{fig:mercator}
\end{figure}

\subsubsection{EPSG:4326 -- WGS 84}
The most recent version of the Mercator projection is called WGS 84. It improves in accuracy compared to WGS 72 and is an \enquote{European Petroleum Survey Group} (EPSG) standard called EPSG:4326, created by \cite{Decker1986}.\\
WGS 84 is an ellipsoidal coordinate system that shows the 3D surface of the earth in 2D. The coordinates are longitude and latitude measured in degree. Longitude has the 0° point in Greenwich, England and increases east to a maximum of 180° and west to a minimum of -180°. Latitude has the 0° point at the equator and increases north to a maximum of 90° and south to a minimum of -90.\\
WGS 84 is used by the Global Positioning System (GPS), GIS specialists, inside the OpenStreetMap (OSM) database, as well as Google Earth.

\subsubsection{EPSG:3857 -- WGS 84 / Pseudo-Mercator}
Pseudo-Mercator is a projected version of WGS 84 into a two-dimensional cartesian coordinate system. It is also referred to as EPSG:3857 and was created by \cite{Grafarend1995}.\\
The CRS is based on a plane, rather than an ellipse. The zero points is identical to WGS 84 but the coordinates are X and Y measured in meters. The Y coordinate is limited to ±85.06° of the WGS 84 bounds. This results in a square projection with a range of ±20,026,376.39m on both axis, but sacrifices the poles to some extend. The square shape allow the creation of tile pyramids, also called \enquote{Mercator Pyramids} to be used for maps in browsers for the web. \\
A pyramid consists of square images in fixed size, e.g. 256x256px. The First level has one image that shows the whole area of the data with minimal detail. On every level below the number of image tiles increases and therefore the level of detail. The number of tiles on a given level is
$$x_n= 4* x_{n-1}$$ 
where \enquote{x} is the number of tiles and \enquote{n} is the current level of detail. Figure \ref{img:mercator-pyramid} shows an example of a Mercator pyramid.
\begin{figure}[H]
	\centering
	\includegraphics[width=0.4\columnwidth]{res/mercator-pyramid}\\
	\caption[]{Mercator tile pyramid. \url{http://data.webglearth.com/doc/webgl-earthch1.html}}
	\label{img:mercator-pyramid}
\end{figure}
This fragmentation of a big dataset is optimal to be loaded on demand and cached by a web browser, which is why major map sites, like OpenStreetMap, Google Maps and Bing Maps use Pseudo-Mercator as their reference system.


\subsection{Spatial Data Types}
Geo-spatial data exist in two different types, vector and raster. 

\subsubsection{Raster Data}
 Raster data consists of a grid in fixed size. Depending on the file, each pixel contains a color (e.g. satellite data)  or a grayscale (e.g. elevation maps) value.\\
 The advantage of raster data is that it is better suited to store picture like data with gradients. Due to its nature, raster has data in every cell, which leads into a high amount of data that has to be stored, so the potential file size if big.\\
 Since the raster size is fixed, the balance between desired detail and small file size has to be decided upon file creation. Figure \ref{img:raster} shows a shape in three different rasterized resolutions, visualizing the loss of detail.
 \begin{figure}[H]
 	\centering\includegraphics[width=.5\textwidth]{res/Vector-Raster}
 	\caption{Shape as raster in different resolutions. \url{http://desktop.arcgis.com/en/arcmap/latest/manage-data/raster-and-images/what-is-raster-data.htm}}
 	\label{img:raster}
 \end{figure}
Common file types for raster data are GeoTIFF, ESRI, ESRI ArcGrid and ASCIIGrid. GeoTIFF and ArcGrid are binary file types and ASCIIGrid is text-based.

\subsubsection{Vector Data}
In contrary to raster data, vectors files don't map color information to a specific pixel, but define mathematical shapes which are rendered as desired. This allows very efficient storage and it is possible to scale the shapes as desired.\\
Data inside vector GIS can have several layers. Each layer can be of the type points, line or polygon. The data on a layer is called a feature. A point feature has a single coordinate, like a well. lines are open polygon lines and are often used for rivers. Shapes are closed lines and are used for areas like countries, lakes, or larger objects. Figure \ref{img:vector} shows a vector file with 3 layers. The point layer shows the position of wells, the linestring layer shows a river and the polygon layer a lake.\\
\begin{figure}[H] 
	\centering
	\includegraphics[width=0.4\columnwidth]{res/vector-map}\\
	\caption[]{Vector map with point, line and polygon. \url{https://commons.wikimedia.org/w/index.php?curid=3024482}}
	\label{img:vector}
\end{figure}

Vector GIS is however limited in its capabilities. The mathematical functions mentioned above are not well suited for data without discrete values. Anything that has gradients is better suited for raster.\\
The most common vector file formats are ESRI Shapefile as binary and GeoJSON as a text-based format.



\section{GIS Technologies}
Chapter \ref{sec:software_design} and \ref{sec:implementation} discuss technologies that were considered at one point of this work. The following section introduces them briefly.


% Performance tests hier???
\subsection{Spatial Databases}
Spatial databases are DBs that have been extended to store and query spatial data types.

\subsubsection{PostGIS}
PostGIS \citep{Obe2011} is an extended PostgreSQL that allows to store and query both raster and vector GIS.

\subsection{Spatial Libraries}



\section{MARS}
MARS (Multi-Agent Research and Simulation) is a working group at the HAW (University of applied Sciences) in Hamburg, Germany. The research group develops an distributed agent-based simulation system to be used by domain experts around the world. It consists of several components.


\subsection{LIFE}
The simulation system is called MARS LIFE. It executes simulation runs created inside the Cloud Services. For more detail see \cite{Huning2016}.


%\subsection{Cloud Services}
%The Cloud Services are the central back-end of the MARS system. These components are responsible for data import, persistence, data visualization an the connection between LIFE and the WebUI.
%
%
%\subsection{WebUI}
%The WebUI is the front-end of MARS. It is a web-based application that allows the user to control back-end services and trigger simulations from the web-browser.


%\section{Mono}
%Open C\# implementation for Linux and OSX.
%
%
%
%\section{.NET Core}
%Microsoft bought Xamarin, which mostly developed Mono and is creating it's new C\# with build in multi-platform support. % Groundwork
	% !TeX spellcheck = en_US

% Requirements
% muss schnell sein
% GIS Layer (Vector & Raster) [Ergebnis der Requirements]
\chapter{Analysis}
This work focuses on providing geospatial capabilities during simulations inside MARS LIFE. To ensure the practical use to the model developers, use-cases were taken into account to generate requirements for the implementation.



\section{Use-Cases}
The following use-cases all regard the LIFE system and therefore share the following preconditions:
\begin{itemize}
	\item The required input data was imported prior to the simulation.
	\item The file has been mapped to the appropriate layer.
	\item The data is available during the simulation.
	\item The file is a recognized and valid GIS file.
\end{itemize}

%\begin{usecase}
%	\addtitle{Vector I}{Open Vector file for reading}
%	\addfield{Summary:}{Open a Vector file that was imported prior to the simulation creation.}
%	\addfield{Preconditions:}{
%%		\item The vector file was imported and mapped prior to the simulation run.
%%		\item The Shapefile is valid (containing at least a .shp, .shx and .dbf file).
%		--
%	}
%	\addfield{Primary Scenario:}{
%		Initiate a vector GIS layer inside the model from a file and enable agents the underlying Shapefile.
%	}
%	%	\additemizedfield{Alternative Scenario:}{
%	%		\item The user clicks the upload button, browses files of his local machine, fills the form for every file and starts the import.
%	%	}
%\end{usecase}

%\begin{usecase}
%	\addtitle{Vector II}{Read all features}
%	\addfield{Summary:}{Read features from a vector file with the following types: 
%		\begin{itemize}[itemsep=-.5em,topsep=.25em]
%			\item Point
%			\item Line
%			\item Polygon
%		\end{itemize}
%	}
%%	\additemizedfield{Preconditions:}{
%%		\item The vector file was imported prior to the simulation run.
%%		\item The vector file is open.
%%	}
%	\additemizedfield{Primary Scenario:}{
%		\item A layer wants to read the positions of all agents during the initialization.
%	}
%\end{usecase}

\begin{usecase}
	\addtitle{Vector I}{Read feature at position}
	\addfield{Summary:}{Read the closest feature to a GPS position.
	}
%	\additemizedfield{Preconditions:}{
%		\item The vector file was imported prior to the simulation run.
%		\item The file is open.
%	}
	\addfield{Primary Scenario:}{
		An agent wants to find a point of interest in his area (e.g. An elephant looking for the closest waterhole).
	}
\end{usecase}

\begin{usecase}
	\addtitle{Vector II}{Calculate distance between features}
	\addfield{Summary:}{A function that takes two features as input and returns the distance.
	}
%	\additemizedfield{Preconditions:}{
%		\item Both features exist and are available to the layer.
%		\item The file is open.
%	}
	\addfield{Primary Scenario:}{
		An agent wants to know the distance to a specific point of interest.
	}
\end{usecase}

\begin{usecase}
	\addtitle{Vector III}{Determine if features intersect}
	\addfield{Summary:}{A functionality that allows to detect, if several features intersect.
	}
%	\additemizedfield{Preconditions:}{
%		\item Both features exist and are available to the layer.
%		\item One Feature has to be of type polygon
%		\item The file is open.
%	}
	\addfield{Primary Scenario:}{
		An agent wants to know the distance to a specific point of interest.
	}
\end{usecase}

%\begin{usecase}
%	\addtitle{Raster I}{Open Raster for reading}
%	\addfield{Summary:}{Open raster that was imported prior to the simulation creation.}
%%	\additemizedfield{Preconditions:}{
%%		\item The file was imported and mapped prior to the simulation run.
%%		\item The file is valid and geo-referenced.
%%	}
%	\additemizedfield{Primary Scenario:}{
%		\item Initiate a raster GIS layer inside the model from a file and enable agents to access it.
%	}
%\end{usecase}

%\begin{usecase}
%	\addtitle{Raster II}{Read all pixels}
%	\addfield{Summary:}{Read all pixels from a raster file supported by the GIS-data-service.
%	}
%%	\additemizedfield{Preconditions:}{
%%		\item The raster file was imported prior to the simulation run.
%%	}
%	\additemizedfield{Primary Scenario:}{
%		\item A layer wants to read the positions of agents during the initialization.
%	}
%\end{usecase}

\begin{usecase}
	\addtitle{Raster I}{Read pixel at position}
	\addfield{Summary:}{Read the closest non no-data pixel on a specific layer to a GPS position.
	}
%	\additemizedfield{Preconditions:}{
%		\item The raster file was imported prior to the simulation run.
%	}
	\addfield{Primary Scenario:}{
		An agent wants to find a point of interest in his area (e.g. An elephant looking for the closest waterhole).
	}
\end{usecase}

\begin{usecase}
	\addtitle{Raster II}{Calculate distance between Pixels}
	\addfield{Summary:}{A function that takes two GPS coordinates as input and returns the distance in Meters.
	}
%	\additemizedfield{Preconditions:}{
%		\item Both features exist and are available to the layer.
%	}
	\addfield{Primary Scenario:}{
		An agent wants to know the distance to a specific point of interest.
	}
\end{usecase}

\section{Requirements}
\label{sec:requirements}
The following requirements were extracted from the use-cases.

\subsection{Functional Requirements}
\reqstartF
%	\item Open vector file.
	\item Read all Features from file into a local data structure.
	\item Read all features at a GPS position.
	\item Calculate distance between features.
	\item Determine if features Intersect.
%	\item Open raster file.
	\item Read all pixels from a raster into a local data structure.
	\item Read a pixel at a specific GPS position.
	\item Read a pixel at a cartesian position.
	\item Calculate distance between Pixels.
	\item Calculate distance between GPS positions.
\reqendF


\subsection{Non-Functional Requirements}
\reqstartNF
	\item .NET Core compatibility
	\item 1,000 agents, each reading a features in parallel takes less than 1 second (1 ms/read).
	\item 1,000 agents, each reading a pixels in parallel takes less than 1 second (1 ms/read).
	\item Integration into MARS LIFE
	\item Errors are handled to prevent simulation crashes.
	\item Log messages are written to stdout.
\reqendNF
 % Analysis
	% !TeX spellcheck = en_US

\chapter{Software Design}
This chapter discusses the possible solutions for working with GIS inside the LIFE simulation system.



\section{Current state}
Currently the GIS support inside MARS is partial, some components of the environment are capable of handling GIS, while others are not currently supporting it.


\subsection{Web browser Front-end}
The WebUI supports GIS for importing and managing this kind of data. It's successor the teaching UI currently does not support GIS. Due to the micro-service structure of the back-end services it is possible for the front-ends to coexist.

\subsection{Back-end Services}
The back-end services fully support GIS. The file service accepts the uploaded files and hands control over to the GIS Data Service (GDS). 

\subsubsection{GIS Data Service}
The GDS is capable of handling the most common GIS types, these are Geotiff and Esri AsciiGrid for raster and Shapefile for vector files. The files may be provided compressed inside a .zip file or as plain files.\\
During the import the GDS determines the type of data automatically by checking the file extension. Once detected, the file is validated and the spatial reference is determined. In case of a valid geo-referenced input, the file is imported into the GeoServer for persistence.


\subsubsection{GeoServer} \label{sec:GS}
The GeoServer (GS) is an open source software that is tailored to store, manage and share GIS. It offers a web GUI as well as a REST API for interaction. The api is structured in a way that it implements common Open Geospatial Consortium (OGC) standards for retrieving data.\\
The Web Feature Service (WFS) allows to retrieve features of vector data, Web Coverage Service (WCS) enables downloading raster data and the Web Map Service (WMS) can generate a tile map for viewing GIS on a map.\\
The GS is build for working with few files and a small number of users. The MARS use-case requires to read thousands to millions of  single values in parallel in a short amount of time. The GS does not satisfy this demands. The retrieval of single values is not supported, since the general use-case is to work with complete files. The performance for retrieving data is also very poor which will require a better solution.


\subsection{GIS Layer}
The GIS Layer inside MARS LIFE reads the Data provided by the GDS and uses it inside the simulation. Since the migration from C\# to .NET Core inside LIFE the GIS layer is not compatible anymore.\\
The current models therefore do not leverage GIS for their input. Data is being transfered into other formats to work around these shortcomings. Vector data is mainly represented as comma separated values (.csv) files and raster data are represented in custom formats.


\section{Solutions}
As mentioned in section \ref{sec:GS} reading data from the GS during the simulation is not feasible inside a high performance environment.\\
Different alternative solutions have been discovered and evaluated. The data could be stored inside a spatial database that is optimized for parallel reads of single values or the GIS file could be supplied to the Simulation and stored in memory for fast reading. Figure \ref{fig:feature_comparison} shows an overview of the evaluated technologies.

\begin{table}[H]
	\centering
	\caption{GIS solutions feature comparison}
	\label{fig:feature_comparison}
	\begin{tabular}{|l|l|l|l|}
	\hline \textbf{Name}	& \textbf{Raster support} & \textbf{Vector support} & \textbf{Type} \\
	\hline GeoServer & yes & yes & self-hosted product \\
	\hline PostGIS	& yes* & yes & Database \\
	\hline MongoDB	& no & yes & Database \\
	\hline Dotspatial & yes & yes & C\# library \\
	\hline NetTopologySuite	& no & yes & C\# library \\
	\hline
	\end{tabular}
\end{table}


\subsection{Performance comparison}
All mentioned technologies were tested with the same input files. A small, a medium and a large size vector and raster file. The file sizes were ~10 KB, ~6.5 MB and ~105 MB. Each test performs 1,000 parallel reads and the result is the average of three separate tests. 
The results are listed in figure \ref{fig:vector_performace} and \ref{fig:raster_performace}

\begin{figure}[H]
	\begin{tikzpicture}
	\begin{axis}[
	xbar,
	y axis line style = { opacity = 0 },
	axis x line       = bottom,
	tickwidth         = 0pt,
	%	enlarge y limits  = 0.02,
	enlarge x limits  = 0.02,
	symbolic y coords = {NetTopologySuite, DotSpatial, MongoDB, PostGis, Geoserver},
	nodes near coords,
	height=9cm,
	legend style={at={(0.8,0.25)},anchor=north},
	xlabel={Time in ms},
	]
	% Small Shapefile
	\addplot coordinates {
		(23103,Geoserver)
		(160,PostGis)
		(135,MongoDB)
		(86,DotSpatial)
		(6,NetTopologySuite)
	};
	% Mid Shapefile
	\addplot coordinates {
		(25307,Geoserver)
		(410,PostGis)
		(168,MongoDB)
		(251,DotSpatial)
		(193,NetTopologySuite)
	};
	% Large Shapefile
	\addplot coordinates {
		(67727,Geoserver)
		(4789,PostGis)
		(223,MongoDB)
		(620,DotSpatial)
		(248,NetTopologySuite)
	};
	\legend{Small Shapefile (12 KB), Medium Shapefile (6.5 MB), Large Shapefile (103.8 MB)}
	\end{axis}
	\end{tikzpicture}
	\caption{Vector performance for 1k reads}
	\label{fig:vector_performace}
\end{figure}


\begin{figure}[H]
	\begin{tikzpicture}
	\begin{axis}[
	xbar,
	y axis line style = { opacity = 0 },
	axis x line       = bottom,
	tickwidth         = 0pt,
	enlarge y limits  = 0.15,
	enlarge x limits  = 0.02,
	symbolic y coords = {DotSpatial, AsciiGrid Parser, PostGis, Geoserver},
	nodes near coords,
	legend style={at={(0.8,0.32)},anchor=north},
	xlabel={Time in ms},
	]
	% Small Rasterfile
	\addplot coordinates {
		(22202,Geoserver)
		(210,PostGis)
		(13,AsciiGrid Parser)
		(4,DotSpatial)
	};
	% Mid Rasterfile
	\addplot coordinates {
		(24438,Geoserver)
		(1964,PostGis)
		(311,AsciiGrid Parser)
		(185,DotSpatial)
	};
	% Large Rasterfile
	\addplot coordinates {
		(26786,Geoserver)
		(nan,PostGis) % 342197
		(467,AsciiGrid Parser)
		(273,DotSpatial)
	};
	\legend{Small raster (8 KB), Medium raster (6.4 MB), Large raster (105.2 MB)}
	\end{axis}
	\end{tikzpicture}
	\caption{Raster performance for 1k reads}
	\label{fig:raster_performace}
\end{figure}


\subsection{PostGIS}
PostGIS is a PostgreSQL with GIS extensions. It supports storing vector data as well as raster data.\\
The performance is way better than the of the GS, however PostGIS did not handle the large file very well. The reading of the large vector file took over 10 times longer than the smaller file, which is surprising since the data is broken into tables. A larger table should not effect the response time by that extend.\\
The big raster file was even worse. The test crashed with an OutOfMemory exception on a machine with 16 GB of RAM. Limiting the parallelism of threads to 7 worked, but the results of around ~340,000 ms (5 min. 40 sec.) were not acceptable.


\subsection{MongoDB}
MongoDB offeres good overall performance for vector data. The file size does not effect the overall performance by a large extend. For the medium and large file it was even faster than NetTopologySuite (NTS). Raster GIS is not supported.


\subsection{DotSpatial}
DotSpatial offers average vector read performance. MongoDB and NTS perform slightly better. Since both these technologies are not available for raster GIS, DotSpatial performes best in that field.\\
As of August 25th 2017 the library has not been ported to .NET Core. This is why an .NET Core compatible Esri AsciiGrid parser was created.\\
The performance measurement for DotSpatial and NTS are different to the database solutions. The way both libraries work is, that they open the file into RAM and read from it. The read performance from RAM for the mentioned scenario is around 0.02 ms for both technologies. Since the file can be loaded once and read from during the simulation, the increased time for additional reads is very low, making these solution preferable.

\subsection{NetTopologySuite}
NTS has the best vector read performance and shares the same RAM based performance benefits as DotSpatial. It is not available for raster GIS. As of August 2017 NTS is available for .NET Core.


\subsection{AsciiGrid Parser}
Since DotSpatial is not ported to .NET Core yet, there is no acceptable solution for reading raster GIS during the simulation. This is, why the author of this work created a parser for Esri's AsciiGrid format.\\
AsciiGrid is a text-based, geo-referenced raster format that other raster formats like GeoTIFF can be converted into. Due to the text-based nature of the format the read performance is not as good as the DotSpatial reader. It is however acceptable until DotSpatial gets migrated.

 % Software Design
	% !TeX spellcheck = en_US

% 
% Lösung beschreiben
% Integration Life
% Performance Test (fertige Implementation)
\chapter{Implementation}
\label{sec:implementation}
This chapter covers the GIS layers integration into the LIFE simulation system, as well as a showcase that validates the practical use.



%\section{Technical limitations}
%Implementing the GIS layer was done in a predefined environment. To meet the given requirements certain limitations had to be overcome.\\
%The biggest one was the usage of .NET Core in combination with the GIS domain. GIS processing is a very specialized field in computing and is used by a small community. The community has a very domain focused background, rather than a technical one. This being said, the community has settled on two main technologies; python and R.\\
%Outside these two languages the number of usable libraries is limited. C\# on the other hand has a big corporate user-base that focuses on big business applications, making it less attractive for people from the GIS domain. As a result library support for C\# is extremely limited. Currently there are only two serious libraries to choose from, namely NetTopologySuite and DotSpatial.\\
%
%
%\subsection{.NET Core}
%.NET Core is a new, implementation of C\# developed by Microsoft. While .NET Core is a very promising technology in terms of performance and platform independence, it still has some shortcomings.\\
%The most relevant issue is the absents of core functionality. Classes like \enquote{System.Drawing} which provides general image manipulation is currently missing in .NET Core v1.1. This particular class has been added to the recently published .NET Core 2.0 which improves the situation. However other functionalities used for GIS are still missing.
%
%
%\subsection{GIS libraries}
%As mentioned before, compatibility for GIS libraries is not very good. Since they rely heavily on missing core functionality, it is not trivial to migrate them and as a result, compatibility to all GIS libraries tested during the project, stopped working with .NET Core.\\
%The only library that has been migrated so far is NetTopologySuite. DotSpatial has not been ported yet and the team did not show any intentions to migrate the code base. This is why the AsciiGrid parser has been created as an alternative (see section \ref{sec:parser}).



\section{Integration into MARS LIFE}
The solutions for raster and vector GIS work quite different, as a result two separate layers have been created. During the initialization phase both layers create an instance for every GIS file of their type. The initialization data contains an URI that allows retrieving the actual file from the GeoServer. Once the file has been retrieved, it is initialized. The process differs between vector and raster data.


\subsection{Vector GIS Layer -- NetTopologySuite}
During the vector layers initialization, the layer opens it's file. Afterwards, the data table is converted into a concurrent dictionary. This allows for fast read and even write access during the simulation in parallel.\\
Once initialized, the layer offers various methods for querying features in parallel. These methods include distance calculation, detecting intersections, reading and writing the data table etc. All the methods are based on the requirements specified in section \ref{sec:requirements}.


\subsection{Raster GIS Layer -- AsciiGridParser}
\label{sec:parser}
Due to the incompatibility of DotSpatial an alternative solution had to be created to handle raster GIS. With the lack of core libraries that support any kind of image processing supporting GeoTIFF was not worth the effort.\\
To do so, image processing, as well as certain image compression mechanism would have had to be implemented. Considering the current design, this effort would have been able to speed up initialization time, but not runtime performance. Therefor the text-based Esri AsciiGrid format has been created.\\
AsciiGrid has larger file size, but zip compression can dramatically reduce this size. The conversion into a zipped AsciiGrid is done by the GeoServer on export.\\
The way the parser works is that it opens the data file during the initialization and parses the metadata into local variables and the payload into a multi-dimensional array. This reduces the complex and time consuming portion of the process to a one time operation. Thereafter, reads can be done via index access on the array, which is almost instant, allows millions of reads within milliseconds.\\
The parser offers methods for a more convenient workflow, as specified in section \ref{sec:requirements}. The easiest way to query the raster file is by X and Y coordinates of the desired pixel. It is however more practical to request based on a GPS position. Additionally it is possible to calculate the distance between pixels and GPS position.\\
The parser is fast and easy to use, since it offers a simple and clean interface that is documented and well tested.


\section{Showcase}
To demonstrate the capabilities of the created GIS Vector layer, it was integrated into MARS LIFE and is used for obstacle detection within the Kruger National Park (KNP) model. The model simulates elephant population for the whole area (19,485 \si{km^{2}}) over 30 years with an hourly resolution.\\
The park is surrounded by a fence that is supposed to keep elephants from leaving the national park. However, in times of desperate need, the elephants are able to break through the fence.


\subsection{Obstacle Layer Replacement}


\subsubsection{Old Implementation}
The old implementation uses a special obstacle layer that just has the purpose to model this kind of behavior. The original Shapefile polygon representing the fence has to be rasterized. The raster is then converted to a custom comma-separated file that represented all columns and lines of the raster image. Each field has either a value of 0 or the strength of the fence at a specific position.\\
The obstacle approach requires extensive preparation and a specialized layer type inside LIFE. The result has the advantage that a preinitialized file is very good in terms of performance.\\
On the other hand, it is not very accurate. The size of a cells was set to a one kilometer by one kilometer, which effectively makes the fence at least 1km thick at any point. The file size of the obstacle input file that had to be uploaded into the MARS cloud was 8.1 MB compared to 38 KB for the original Shapefile.


\subsubsection{GIS  Implementation}
The GIS layer is a generic implementation that allows the usage of vector files. To make the obstacle detection work, a data table with a column named "resistance" and one line, containing the value had to be added. The field holds the strength of the fence.\\
During the elephant movement, it simply requests the accumulated path rating of the desired path. The GIS layer checks if the path intersects with a feature of the GIS file. If this is the case, it reads the resistance value of the feature's data table and adds these value up for every intersection.\\
E.g. no intersection with a fence of strength 19,000 would return 0, one intersection 19,000, two results in 38,000 and so on.


\subsection{Other possible Use-Cases}
The generic nature of the GIS layer enables the model developer to encode any type of table data. E.g. it would be possible to use a Shapefile with polygons that contain water holes. If an elephant is within a certain distance, it could sense the water and move towards it. This mimics the functionality currently provided by the potential-field layer, that has similar constrains compared to the obstacle layer.\\
Another possibility is to represent vegetation. Similar to the previous showcases, this requires areas that hold certain data. Additional these values have to be able to change over time, because vegetation is consumed and grows back. Since the data table is held in RAM inside a data-structure that support parallel access it, this functionality can be used by models without adjustments to the layer.
 % Implementation
	% !TeX spellcheck = en_US

\chapter{End}



\section{Conclusion}
This work described the process of optimizing the GIS read performance inside the MARS LIFE system. The implementation is a rewrite of the original GIS layer with a separated vector and raster GIS layer. The new layers satisfy the requirement of .NET Core compatibility and significantly improve performance compared to the old implementation.


\section{Outlook}
The .NET Core support for DotSpatial might come in the future. Migrating to it could save some memory due to the smaller binary raster GIS files. This is however no critical change since it can be done with minimal effort and no effect to model code relying on the raster GIS layer.\\
Storage of geospatial files inside the GeoServer is debatable. Currently it is simply used for storing and retrieving files. The only functionally used is the conversion of GeoTiff to AsciiGrid. This functionality could be taken over by the GIS-Data-Service that controls the GeoServer. Since files are stored in the filesystem the files could be directly written to the FS of the GDS or stored inside a more lightweight storage system like PostGIS.\\
Management of GIS can be further improved. Functionality that aggregates and correlates data during import is possible. This could combine input data to a common spatial area, save space, further increase performance and help gain information over several input files. % End
	
	
	%%%%%%%%%%%%%%%%%%%%%%%%%%%%%%%%%%%%%%%%%%%%%%%%%%%%%%%%%%%%%%%%%%%%%%%%%%%%%%%%%%
	%% appendix if used
	%%\appendix
	%\typeout{===== File: appendix}
	%% !TeX spellcheck = en_US

\begin{landscape}

\chapter{Appendix}

\begin{table}[H]
	\caption{GIS Technology overview}
	\label{fig:technologies}
	\resizebox{1.1\linewidth}{!}{
	\begin{tabular}{|l|l|l|l|}
		\hline \textbf{Name} & \textbf{Description} & \textbf{Discarded} & \textbf{URL}\\
		\hline AsciiGridParser & Own .NET Core Component & No & --\\
		\hline Dotspatial & C\# library & No & \url{https://github.com/DotSpatial/DotSpatial}\\
		\hline GeoServer & Self-hosted product & No & \url{http://geoserver.org}\\
		\hline MongoDB	& DB with spatial capabilities & No & \url{https://www.mongodb.com}\\
		\hline NetTopologySuite	& C\# library & No & \url{https://github.com/NetTopologySuite/NetTopologySuite}\\
		\hline PostGIS	& PostgreSQL DB + GIS ext. & No & \url{http://postgis.net}\\
		\hline
		\hline Vertica & A proprietary analytic database owned by Hewlett Packard. & Yes & \url{https://www.vertica.com}\\
		\hline HadoopDB & Hybrid of MapReduce and DBMS Technologies for Analytical Workloads. & Yes & \cite{Abouzeid2009}\\
		\hline Hadoop-GIS & A High Performance Spatial Data Warehousing System over MapReduce. & Yes & \cite{Wang2011}\\
		\hline
	\end{tabular}
	}
\end{table}

\end{landscape}
	
	% bibliography and other stuff
	\backmatter
	
	\typeout{===== Section: literature}
	%% read the documentation for customizing the style
	\bibliographystyle{apa}
	\bibliography{hauptprojekt}
	
	\typeout{===== Section: nomenclature}
	%% uncomment if a TOC entry is needed
	%%\addcontentsline{toc}{chapter}{Glossar}
	\renewcommand{\nomname}{Glossar}
	\clearpage
	\markboth{\nomname}{\nomname} %% see nomencl doc, page 9, section 4.1
	\printnomenclature
	
	%% index
	\typeout{===== Section: index}
	\printindex
	
	%\HAWasurency
	
\end{document}