% !TeX spellcheck = en_US

\chapter{Introduction}
Geospatial data plays a major roll in the processing big data and in the age of mobile geo-referenced data is becoming increasingly relevant \citep{Lee2015, Kitchin2013, Graham2013}. With growing relevance of such big data, the efficient storage and manipulation is a crucial topic.\\
MARS LIFE provides a large scale simulation environment for multi-agent simulations. Most models rely on geo-referenced data, making it a natural match for GIS. The possibility to use it as input for simulations is therefor very desirable.\\
This work focuses on improving the LIFE simulation system by adding layers that allows to take advantage of raster and vector GIS capabilities.



\section{Motivation}
MARS LIFE used to have capabilities for using GIS. However, they did not meet the requirements in terms of performance and usability. For this reason they were never used in production. With a recent infrastructural change, it has been abandoned due to incompatibilities.\\
The incapability to use GIS created the necessary for workarounds to use geo-referenced data. While these solutions work, they cover a very narrow use-case tailored to the needs of a specific model. This contradicts the idea of a general purpose simulation system.\\
For the reasons mentioned above this work focuses on re-instantiating the GIS into MARS LIFE and to meet the requirements in terms of performance and usability. The main focus is, to achieve a performance that can handle agent counts in the millions within a reasonable time.
