% !TeX spellcheck = en_US

\chapter{Analysis}
The goal of this work is to create a layer inside the simulation system MARS LIFE that allows loading geo-spatial data in an high-performance way. To ensure the practical use to the model developers, use-cases were taken into account to generate requirements for the implementation.



\section{Use-Cases}
\begin{usecase}
	\addtitle{Vector I}{Open Vector file for reading}
	\addfield{Summary:}{Open Shapefile that was imported prior to the simulation creation.}
	\additemizedfield{Preconditions:}{
		\item The vector file was imported and mapped prior to the simulation run.
		\item The Shapefile is valid (containing at least a .shp, .shx and .dbf file).
	}
	\additemizedfield{Primary Scenario:}{
		\item Initiate a vector GIS layer inside the model from a file and enable agents to access it.
	}
	%	\additemizedfield{Alternative Scenario:}{
	%		\item The user clicks the upload button, browses files of his local machine, fills the form for every file and starts the import.
	%	}
\end{usecase}

\begin{usecase}
	\addtitle{Vector II}{Read all features}
	\addfield{Summary:}{Read features from a vector file with the following types: 
		\begin{itemize}[itemsep=-.5em,topsep=.25em]
			\item Point
			\item Line
			\item Polygon
		\end{itemize}
	}
	\additemizedfield{Preconditions:}{
		\item The vector file was imported prior to the simulation run.
		\item The vector file is open.
	}
	\additemizedfield{Primary Scenario:}{
		\item A layer wants to read the positions of all agents during the initialization.
	}
\end{usecase}

\begin{usecase}
	\addtitle{Vector III}{Read feature at position}
	\addfield{Summary:}{Read the closest feature to a GPS position.
	}
	\additemizedfield{Preconditions:}{
		\item The vector file was imported prior to the simulation run.
		\item The file is open.
	}
	\additemizedfield{Primary Scenario:}{
		\item An agent wants to find a point of interest in his area (e.g. An elephant looking for the closest waterhole).
	}
\end{usecase}

\begin{usecase}
	\addtitle{Vector IV}{Calculate distance between features}
	\addfield{Summary:}{A function that takes two features as input and returns the distance in meters.
	}
	\additemizedfield{Preconditions:}{
		\item Both features exist and are available to the layer.
		\item The file is open.
	}
	\additemizedfield{Primary Scenario:}{
		\item An agent wants to know the distance to a specific point of interest.
	}
\end{usecase}

\begin{usecase}
	\addtitle{Vector V}{Calculate if feature overlaps another}
	\addfield{Summary:}{A function that takes two features as input and returns true if they overlap.
	}
	\additemizedfield{Preconditions:}{
		\item Both features exist and are available to the layer.
		\item One Feature has to be of type polygon
		\item The file is open.
	}
	\additemizedfield{Primary Scenario:}{
		\item An agent wants to know the distance to a specific point of interest.
	}
\end{usecase}


\begin{usecase}
	\addtitle{Raster I}{Open Raster for reading}
	\addfield{Summary:}{Open raster that was imported prior to the simulation creation.}
	\additemizedfield{Preconditions:}{
		\item The file was imported and mapped prior to the simulation run.
		\item The file is valid and geo-referenced.
	}
	\additemizedfield{Primary Scenario:}{
		\item Initiate a raster GIS layer inside the model from a file and enable agents to access it.
	}
\end{usecase}


\begin{usecase}
	\addtitle{Raster II}{Read all pixels}
	\addfield{Summary:}{Read all pixels from a raster file supported by the GIS-data-service.
	}
	\additemizedfield{Preconditions:}{
		\item The raster file was imported prior to the simulation run.
	}
	\additemizedfield{Primary Scenario:}{
		\item A layer wants to read the positions of agents during the initialization.
	}
\end{usecase}

\begin{usecase}
	\addtitle{Raster III}{Read pixel at position}
	\addfield{Summary:}{Read the closest non no-data pixel on a specific layer to a GPS position.
	}
	\additemizedfield{Preconditions:}{
		\item The raster file was imported prior to the simulation run.
	}
	\additemizedfield{Primary Scenario:}{
		\item An agent wants to find a point of interest in his area (e.g. An elephant looking for the closest waterhole).
	}
\end{usecase}

\begin{usecase}
	\addtitle{Raster IV}{Calculate distance between Pixels}
	\addfield{Summary:}{A function that takes two GPS coordinates as input and returns the distance in kilometers.
	}
	\additemizedfield{Preconditions:}{
		\item Both features exist and are available to the layer.
	}
	\additemizedfield{Primary Scenario:}{
		\item An agent wants to know the distance to a specific point of interest.
	}
\end{usecase}

\section{Requirements}
The following requirements were extracted from the use-cases.

\subsection{Functional Requirements}
\reqstartF
	\item Open vector file.
	\item Read all Features from file into a local data structure.
	\item Read all features at a GPS position.
	\item Calculate distance between features.
	\item Determine if features overlap.\\
	\item Open raster file.
	\item Read all pixels from a raster into a local data structure.
	\item Read a pixel at a specific GPS position.
	\item Read a pixel at a cartesian position.
	\item Calculate distance between Pixels.
	\item Calculate distance between GPS positions.
\reqendF


\subsection{Non-Functional Requirements}
\reqstartNF
	\item 10,000 agents, each reading a features in parallel takes less than 10 seconds (1 ms per read).
	\item 10,000 agents, each reading a pixels in parallel takes less than 10 seconds (1 ms per read).
	\item Errors are handled to prevent simulation crashes.
	\item Log messages are written to stdout.
\reqendNF
