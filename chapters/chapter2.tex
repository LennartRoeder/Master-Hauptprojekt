% !TeX spellcheck = en_US

\chapter{Related Work}
This chapter elaborates on the base technologies necessary to understand the following chapters.



\section{Geographic Information System (GIS)}
GIS contains of numerous technologies to store, manipulate, analyze and visualize geographical data. It can be used in all areas, that need to visualize spatial data. Some usages are: land-use, elevation data, weather maps, street networks and flood maps. To leverage the capabilities that GIS offers it is necessary to use specialized GIS software.


\subsection{Coordinate Reference System (CRS)}
In comparison to normal image data, GIS is always geo-referenced. This is achieved by specific geo-aware formats that encode spatial positions into the data. \\
When storing GIS, it is necessary to translate the data into coordinate system. In GIS terminology this is referred to as \enquote{spatial reference system} (SRS) or \enquote{coordinate reference system} CRS. Depending on the spatial area, different coordinate systems vary in their results. Some are optimized for certain areas, while others offer a general worldwide view.

\subsubsection{Mercator Projection}
The most globally used CRS is the Mercator projection. I was created by Gerardus Mercator in 1569 and has been improved over the years. Figure \ref{fig:mercator} shows the Mercator projection in it's normal and transverse orientation. The normal projection offers good general representation, while the transverse orientation is focuses on the poles.
\begin{figure}[H]
	\centering\includegraphics[width=1\textwidth]{res/Mercator}
	\caption{Normal and transverse Mercator projection by Peter Mercator. \url{https://commons.wikimedia.org/w/index.php?curid=9910866}}
	\label{fig:mercator}
\end{figure}

\subsubsection{EPSG:4326 -- WGS 84}
The most recent version of the Mercator projection was created 1984 and is an \enquote{European Petroleum Survey Group} (EPSG) standard called EPSG:4326 by \cite{Decker1986}.\\
WGS 84 is an ellipsoidal coordinate system that shows the 3D surface of the earth in 2D. The coordinates are longitude and latitude measured in degree. Longitude has the 0° point in Greenwich, England and increases east to a maximum of 180° and west to a minimum of -180°. Latitude has the 0° point at the equator and increases north to a maximum of 90° and south to a minimum of -90.\\
WGS 84 is used by the Global Positioning System (GPS), GIS enthusiasts, inside the OpenStreetMap (OSM) database, as well as Google Earth.

\subsubsection{EPSG:3857 -- WGS 84 / Pseudo-Mercator}
Pseudo-Mercator is a projected version of WGS 84 into a cartesian coordinate system. The EPSG Standard is 3857 (EPSG:3857) by \cite{Grafarend1995}.\\
The CRS is based on a plane, rather then an ellipse. The center is identical to WGS 84 but the coordinates are X and Y measured in meters. The Y coordinate is limited to ±85.06° of the WGS 84 bounds. This results in a square projection with a range of ±20,026,376.39m on both axis. This shape allow the creation of tile pyramids, also called \enquote{Mercator Pyramids}. \\
The pyramid is created, by cutting images of fixed size, usually 256px in width and height. Starting at one tile on the first zoom level, the amount of tiles is multiplied by four on every additional zoom level, to a maximum of around 18. Figure \ref{img:mercator-pyramid} shows such a tile pyramid.\\
The resulting small images are ideally to be loaded on demand by a web browser. This is, why all major map sites, like OSM, Google Maps and Bing Maps use Pseudo-Mercator as their CRS.
\begin{figure}[H]
	\centering
	\includegraphics[width=0.4\columnwidth]{res/mercator-pyramid.png}\\
	\caption[]{Mercator Pyramide}
	\label{img:mercator-pyramid}
\end{figure}

\subsection{Spatial Data Types}
GIS files exist in two different types, vector and raster.

\subsubsection{Raster Data}
 Raster data consists of a grid that is filled with color values. Each pixel stores information as a numeric value. These values can be numeric (e.g. for an elevation map) as well as encoded grayscale or color values.\\
 The advantage of raster data is that it can store any kind of GIS. the disadvantages are big files and the fixed pixel resolution. Figure \ref{fig:vector-raster} shows a shape in three different raster resolutions.
 
 \begin{figure}[H]
 	\centering\includegraphics[width=.5\textwidth]{res/Vector-Raster}
 	\caption{Shape as raster in different resolutions. \url{http://desktop.arcgis.com/en/arcmap/latest/manage-data/raster-and-images/what-is-raster-data.htm}}
 	\label{fig:vector-raster}
 \end{figure}

\subsubsection{Vector Data}
Vector GIS on the other hand has a much smaller file size, but is limited in its capabilities. It stores information based on mathematical functions. While this is very efficient, it is not suited for image like data such as land-use or elevation maps.\\
Data inside vector GIS is called a feature. Features are composed of either points, lines or polygons. It is well suited to store data such as hot-spots, borders, contour lines, etc. 



%\section{Mono}
%Open C\# implementation for Linux and OSX.
%
%
%
%\section{.NET Core}
%Microsoft bought Xamarin, which mostly developed Mono and is creating it's new C\# with build in multi-platform support.